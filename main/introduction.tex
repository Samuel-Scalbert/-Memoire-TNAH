\chapter*{Introduction}
\addcontentsline{toc}{chapter}{Introduction}

La transformation numérique des données extraites de collections d'archives, ainsi que leur diffusion au public sous forme de documents numériques dans divers formats et/ou sous forme d'éditions en ligne, représentent un défi majeur dans le domaine de la recherche et de la préservation du patrimoine culturel. Le projet DAHN, fruit d'une collaboration technologique et scientifique entre l'Inria, l'Université du Mans et l'École des hautes études en sciences sociales (EHESS), financé par le ministère français de l'Enseignement supérieur, de la Recherche et de l'Innovation, s'est engagé à relever ce défi.

DiScholEd (\textit{Digital Scholarly Editions}) a alors été créée pour la mise en ligne des documents numériques. Le projet, en septembre 2023, compte sept corpus. Pour la publication des résultats de la \textit{pipeline} du projet, le choix a été fait d'utiliser un CMS (Content Management System) spécialisé dans les éditions numériques : TEI Publisher.

TEI Publisher est un CMS qui vise à permettre à des chercheurs n'ayant pas de compétences en informatique de publier leurs éditions numériques en ligne. DiScholEd ne dispose pas de développeur attitré au projet, il était donc nécessaire d'utiliser un CMS qui utilise le \textit{low-code} tel que TEI Publisher.

Le \textit{low-code} pour les CMS est une approche de développement web qui repose sur l'utilisation de plateformes ou d'outils générant automatiquement une partie du code nécessaire pour créer des sites web ou des applications web. Contrairement au \textit{no-code} (une approche qui empêche de coder directement l'application), le \textit{low-code} permet généralement un niveau de personnalisation et de contrôle plus élevé en permettant aux utilisateurs de modifier le code généré ou d'ajouter des éléments personnalisés si nécessaire. Cela rend le \textit{low-code} adapté aux projets nécessitant une plus grande flexibilité tout en réduisant la quantité de code à écrire manuellement, ce qui accélère le processus de développement.

Dans son article sur les éditions scientifiques numériques, Elena Pierazzo soulève des questions cruciales sur l'utilisation des outils numériques et de la recherche textuelle\footcite{pierazzo:hal-02117714}. Elle se demande si les chercheurs doivent être en mesure d'utiliser ces outils eux-mêmes, si une assistance est nécessaire pour la configuration des éditions et qui devrait fournir cette assistance ?

Notre mémoire vise à réaliser une étude de cas autour de DiScholEd et de TEI Publisher. L'objectif est d'analyser si TEI Publisher est adapté à notre projet et si le \textit{low-code} nous permet de créer notre application sans l'aide d'un développeur tout en produisant une application originale.

Ce mémoire est structuré en trois parties. Dans la première partie, nous présenterons le contexte du projet DiScholEd et de ses corpus, ainsi que les caractéristiques essentielles des CMS, en mettant particulièrement l'accent sur TEI Publisher. Nous aborderons également la gestion des erreurs dans TEI Publisher.

La deuxième partie se penchera sur les avantages et les inconvénients de TEI Publisher, basés sur les travaux réalisés pendant le stage. Nous analyserons trois aspects cruciaux qui orientent le choix d'un CMS lors du développement d'une application web : la gestion des erreurs, la personnalisation des fonctionnalités et la possibilité de créer un design propre à notre application.

Enfin, la troisième partie explorera les perspectives d'amélioration de TEI Publisher et de l'application DiScholEd. Nous évaluerons l'adéquation de TEI Publisher pour différents types d'utilisateurs en mettant en évidence les défis liés à l'équilibre entre facilité d'utilisation et personnalisation pour les chercheurs. Nous proposerons ensuite des améliorations potentielles pour TEI Publisher et discuterons des évolutions futures possibles pour l'application DiScholEd.