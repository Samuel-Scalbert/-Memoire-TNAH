\chapter*{Conclusion}
\addcontentsline{toc}{chapter}{Conclusion}

Ce mémoire a pour objectif de présenter le travail accompli au cours du stage, ainsi que de réfléchir aux causes des problèmes rencontrés, à l'implémentation de nouvelles fonctionnalités et au \textit{design} de notre application. Cette réflexion nous a permis de mieux appréhender les avantages et inconvénients de l'utilisation de TEI Publisher.

TEI Publisher propose des fonctionnalités avancées pour la gestion de documents XML. Grâce à une approche en \textit{low-code}, il offre la possibilité à des chercheurs peu familiarisés avec l'informatique de publier leurs éditions de manière aisée et rapide, et permet aussi aux développeurs de créer des applications complexes. Mais pour les chercheurs, les restrictions qu'impose TEI Publisher vont rapidement imposer : soit de recruter un développeur pour créer une application spécifique, ou alors de se contenter d'une application simple et fonctionnelle mais limitée dans le \textit{design}. Pour certains développeurs, ils préféreront créer leur propre application.

En privilégiant l'accessibilité pour tous, TEI Publisher se révèle être une solution pertinente pour les éditions numériques. Néanmoins, à travers l'analyse de cas d'études et les correctifs apportés à l'application, il apparaît que de nombreux problèmes subsistent. Ces problèmes ne sont pas inhérents au CMS lui-même, mais résultent d'erreurs dans l'encodage des documents XML. Heureusement, nous avons réussi à les résoudre tous sans compromettre l'intégrité de l'application. La correction de ces problèmes a également permis de mieux comprendre leur origine et d'envisager une solution pour les prévenir : l'utilisation de GitHub Actions.

DiScholEd est une application originale qui cherche à rendre accessible sept corpus malgré leurs différences, ce qui soulève de nouveaux défis et met en lumière les limitations de TEI Publisher. Il est ardu, voire impossible, pour un CMS de s'adapter à tous les projets. Bien que TEI Publisher aspire à être accessible et ouvert à tous, s'il reste un outil utile et novateur pour les éditions numériques, qui n’est cependant pas exempt de faiblesses.